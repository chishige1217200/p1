\documentclass[a4j,11pt]{jarticle}
% ファイル先頭から\begin{document}までの内容(プレアンブル)については,
% 教員からの指示がない限り, { } の中を書き換えるだけでよい.

% この部分は変更しないこと!
\usepackage[top=25truemm,  bottom=30truemm,
            left=25truemm, right=25truemm]{geometry}

% ToDo: 提出要領に従って,適切なタイトル・サブタイトルを設定する
% 「第 X 回」の表記は作成の都度変更すること.
\title{工学基礎実験実習 \\
       ファイル操作とシェル 第 2 回レポート}

% ToDo: 自分自身の氏名と学生番号に書き換える
\author{氏名: 重近 大智 (SHIGECHIKA, Daichi) \\
        学生番号: 09501527}

% ToDo: 教員の指示に従って適切に書き換える
\date{出題日: 2019年5月21日 \\
      提出日: 2019年5月日 \\
      締切日: 2019年5月28日 \\}  % 注:最後の\\は不要に見えるが必要.

% ToDo: 図を入れる場合,以下の1行を有効にする
%\usepackage{graphicx}

\begin{document}
\maketitle
% 目次つきの表紙ページにする場合はコメントを外す
%{\footnotesize \tableofcontents \newpage}

%==================================================================%
\section{はじめに}
ファイルリダイレクションとパイプ,シェルスクリプトやコマンドについて記述する.
機能及び具体例についても記述する.
%==================================================================%
\section{ファイルリダイレクションとパイプ}
 \subsection{リダイレクション}
\subsubsection{標準出力のリダイレクション}
   \begin{description}
\item[機能]
一般的にコマンドは実行すると端末上に結果が表示されるが,\verb|>|記号を使うことで,ファイルに結果を出力することができる. 具体的には以下のようにする.
\item[具体例]
\verb|cat|コマンドによって出力される「solarsystem1」の表示結果を「datalist」に書き込む場合を考える.
\begin{verbatim}
$ cat solarsystem1 > datalist
$ cat datalist
Sun      0.000  1304000      1.41
Mercury  0.579        0.056  5.43
Venus    1.082        0.857  5.24
Earth    1.496        1.000  5.52
Mars     2.279        0.151  3.93
$
\end{verbatim}
上記のようにすると,\verb|cat|によって出力された「solarsystem1」の内容が「datalist」に出力される.
   \end{description}

\subsubsection{標準入力のリダイレクション}
\begin{description}
\item[機能]
逆に、\verb|<|記号を用いることで,データなどをファイルから入力することができる. 具体的には以下のようにする.
\item[具体例]
\verb|sort|コマンドの並びかえの対象を「solarsystem1」から入力し,2番目の項目を降順に並べる場合を考える.
\begin{verbatim}
$ sort -k 2 -nr < solarsystem1
Mars     2.279        0.151  3.93
Earth    1.496        1.000  5.52
Venus    1.082        0.857  5.24
Mercury  0.579        0.056  5.43
Sun      0.000  1304000      1.41
$ 
\end{verbatim}
上記のようにすると,\verb|sort|コマンドの並びかえの対象を「solarsystem1」から入力することができる.
\end{description}

\subsection{パイプ}
\begin{description}
  \item[機能]
あるコマンドの出力結果を別のコマンドに入力する場合,別のファイルを経由せず直接入力することができる. この時にパイプと呼ばれる機能を利用する.
  \item[具体例]
「solarsystem1」というファイルの表示結果を\verb|sort|コマンドの入力にする場合を考える.
    \begin{verbatim}
$ cat solarsystem1 | sort -k 4 -nr
Earth    1.496        1.000  5.52
Mercury  0.579        0.056  5.43
Venus    1.082        0.857  5.24
Mars     2.279        0.151  3.93
Sun      0.000  1304000      1.41
$
    \end{verbatim}
上記のようにすると,\verb|cat|コマンドによって出力された「solarsystem1」というファイルの表示結果が,\verb|sort|コマンドに入力される.
\end{description}

\section{コマンドと実行可能ファイルの関係}
\subsection{ソースファイルの作成}
\begin{description}
\item「Windows7」と表示するC言語のプログラムのソースファイルを作ることを考える.
ファイル名は「Windows7.c」とし,内容は以下の通りである.
\begin{verbatim}
  1	#include<stdio.h>
     2	int main(void)
     3	{
     4	    printf ("Windows7\n");
     5	    return 0;
     6	}
\end{verbatim}
\end{description}
\subsection{ソースファイルのコンパイルと実行可能ファイルの実行}
\begin{description}
\item gccコマンドを用いてコンパイルすると,実行可能ファイル「a.out」が作成され,これを端末上で実行すると以下のようになる.
\begin{verbatim}
$ gcc Windows7.c
$ ./a.out
Windows7
\end{verbatim}
\verb|./a.out|を実行すると,結果として「Windows7」が表示される.
\end{description}

\subsection{シェルスクリプト}
実行可能形式のファイルは,C言語のソースファイルをコンパイルするだけでなく,シェルスクリプトによって作成することもできる.
ただしシェルスクリプトを実行するには,事前に\verb|chmod|コマンドを用いて実行パーミッションを付加する必要がある.

ここでは実行すると「Windows10」と表示するシェルスクリプト「Windows10.sh」を作成する.
スクリプトは以下の通りである.
\begin{verbatim}
#!/bin/sh
echo "Windows10"
\end{verbatim}
実行パーミッションを付加するには,次のように入力する.
\begin{verbatim}
$ chmod 755 Windows10.sh
\end{verbatim}
C言語の実行可能ファイル「a.out」と同様にシェルスクリプトを実行する.
実行するには,次のように入力する.
\begin{verbatim}
$ ./Windows10.sh
Windows10
\end{verbatim}
\verb|./Windows10.sh|を実行すると,結果として「Windows10」が表示される.

\section{その他のコマンド}
ここでは前回のレポートに記述しなかったいくつかのコマンドを紹介する.今回はオプションの確認に\verb|man|コマンドも用いた.
\subsection{\texttt{ps}コマンド}
\verb|ps|コマンドの概要は,以下の通りである.
\begin{description}
  \item[機能]
    実行中のプロセスを表示する.
  \item[形式]
    \verb|ps| \verb|(option)|
  \item[オプション]
    \begin{itemize}
      \item \verb|-e|:  システム上の全てのプロセスを表示する.
      \item \verb|-u|:  rootとして実行されている全てのプロセスを表示する.
    \end{itemize}
  \item[使用例]
    \begin{verbatim}
$ ps
  PID TTY          TIME CMD
 4064 pts/0    00:00:41 emacs
24252 pts/0    00:00:00 ps
31768 pts/0    00:00:00 bash
32556 pts/0    00:00:22 evince
$ 
    \end{verbatim}
\verb|ps|コマンドを実行すると,上記のような結果が得られた.
この結果より,エディタ(emacs),\verb|ps|コマンド,\verb|evince|コマンドなどが
実行されていることが分かる.
\verb|ps -l|とすると,次のような詳細な表示になった.
\begin{verbatim}
$ ps -l
F S   UID   PID  PPID  C PRI  NI ADDR SZ WCHAN  TTY          TIME CMD
0 S  3327  4064 31768  0  80   0 - 288841 poll_s pts/0   00:00:42 emacs
4 R  3327 31162 31768  0  80   0 - 38304 -      pts/0    00:00:00 ps
0 S  3327 31768 31756  0  80   0 - 28758 do_wai pts/0    00:00:00 bash
0 S  3327 32556 31768  0  80   0 - 359568 poll_s pts/0   00:00:23 evince
$ 
\end{verbatim}
\end{description}

% which p.58の4.3節,&,^Z,bg,killはp.59の4.4節





% 5.aliasの使用










\subsection{\texttt{rmdir}コマンド}
\verb|rmdir|コマンドの概要は,以下の通りである.
\begin{description}
  \item[機能] % コマンドの機能内容を書く.
    ディレクトリを削除する.
  \item[形式] % コマンドの入力形式を書く.
    \verb|rmdir| \verb|(option)| \verb|[directory name]|
  \item[オプション] % オプションと内容を箇条書きで書く.
    \begin{itemize}
      \item \verb|--ignore-fail-on-non-empty|:  ディレクトリが空でないため削除に失敗した場合,エラーだけを無視する.
    \end{itemize}
  \item[使用例] % コマンドの実行時に得られる結果を示す.
    \begin{verbatim}
    $ rmdir 1
      $
    \end{verbatim}
\end{description}

\subsection{\texttt{cd}コマンド}
\verb|cd|コマンドの概要は,以下の通りである.
\begin{description}
  \item[機能] % コマンドの機能内容を書く.
    カレントディレクトリを変更する.
  \item[形式] % コマンドの入力形式を書く.
    \verb|cd| \verb|[directory name]|
  \item[オプション] % オプションと内容を箇条書きで書く.
    \verb|[directory name]|を入力しなければ,ホームディレクトリに戻る.
  \item[使用例] % コマンドの実行時に得られる結果を示す.
    \begin{verbatim}
    $ cd p1
      $
    \end{verbatim}
\end{description}
%------------------------------------------------------------------%==================================================================%
\section{ファイル操作のコマンド}

%-----------------------------------------------------------------%
\subsection{\texttt{cat}コマンド}
\verb|cat|コマンドの概要は,以下の通りである.
\begin{description}
  \item[機能] % コマンドの機能内容を書く.
    ファイル内容を連結したり,表示したりする.
  \item[形式] % コマンドの入力形式を書く.
    \verb|cat| \verb|(option)| \verb|[directory name]|
  \item[オプション] % オプションと内容を箇条書きで書く.
オプションは下記の通りである.
 \begin{itemize}
      \item \verb|-n|:  行番号を付けて表示する.
      \item \verb|-v|:  制御コードなどを含むファイルを表示する時に指定する.
      \item \verb|-b|:  空行を除いて行番号を付け加える. \verb|-n|より優先される.
    \end{itemize}
  \item[使用例] % コマンドの実行時に得られる結果を示す.
    \begin{verbatim}
$ cat -n bindec.c
     1	main(){
     2	  int i,j,k,l,p=0;
     3	  printf("2進数\t10進数\n");
     4	  for(i=0;i<2;i++)
     5	    for(j=0;j<2;j++)
     6	      for(k=0;k<2;k++)
     7		for(l=0;l<2;l++){
     8		  printf("%d%d%d%d\t%d\n",i,j,k,l,p++);
     9		}
    10	}

    \end{verbatim}
\end{description}

\subsection{\texttt{less}コマンド}
\verb|less|コマンドの概要は,以下の通りである.
\begin{description}
  \item[機能] % コマンドの機能内容を書く.
    ファイル内容を1画面ごとに表示する.
  \item[形式] % コマンドの入力形式を書く.
    \verb|less| \verb|[directory name]|
  \item[オプション] % オプションと内容を箇条書きで書く.
オプションは下記の通りである.ただし,ファイルを開いた状態でのみ使用できる.
 \begin{itemize}
      \item \verb|e|:  1行進む.
      \item \verb|y|:  1行戻る.
      \item \verb|f|:  1画面進む. 
      \item \verb|b|:  1画面戻る.
    \end{itemize}
  \item[使用例] % コマンドの実行時に得られる結果を示す.
    \begin{verbatim}
$ less report1_09501527.tex
    \end{verbatim}
\end{description}

\subsection{\texttt{mv}コマンド}
\verb|mv|コマンドの概要は,以下の通りである.
\begin{description}
  \item[機能] % コマンドの機能内容を書く.
     ファイルを移動したり,ファイルの名前を変更したりする.
  \item[形式] % コマンドの入力形式を書く.
    \verb|mv| \verb|(option)| \verb|[directory name]| \verb|[directory name]|
  \item[オプション] % オプションと内容を箇条書きで書く.
オプションは下記の通りである.ただし,\verb|-i|,\verb|-f|,\verb|-n|を一つ以上使用した場合は最後のオプションが使用される.
\begin{itemize}
      \item \verb|-i|:  上書きの前に確認を行う.
      \item \verb|-f|:  上書きの前に確認を行わない.
      \item \verb|-n|:  既存のファイルには上書きしない.
      \item \verb|--backup|:  上書き前にバックアップを作成する.
    \end{itemize}
  \item[使用例] % コマンドの実行時に得られる結果を示す.
    \begin{verbatim}
$ mv -i a.txt b.txt
mv: `b.txt' を上書きしますか? y
  $ 
    \end{verbatim}
\end{description}

\subsection{\texttt{cp}コマンド}
\verb|cp|コマンドの概要は,以下の通りである.
\begin{description}
  \item[機能] % コマンドの機能内容を書く.
     ファイルを移動したり,ファイルの名前を変更したりする.
  \item[形式] % コマンドの入力形式を書く.
    \verb|cp| \verb|(option)| \verb|[directory name]| \verb|[directory name]|
  \item[オプション] % オプションと内容を箇条書きで書く.
オプションは下記の通りである.
 \begin{itemize}
      \item \verb|-l|:  コピーの代わりにファイルのハードリンクを作成する.
      \item \verb|-a|:  ファイルの属性のみコピーする.
      \item \verb|--backup|:  上書き前にバックアップを作成する.
    \end{itemize}
  \item[使用例] % コマンドの実行時に得られる結果を示す.
    \begin{verbatim}
$ cp a.txt p1
  $
    \end{verbatim}
\end{description}

%------------------------------------------------------------------


%==================================================================%
\section{考察}
本レポートの作成を通して,CentOSの端末下におけるいくつかのコマンドの使い方,オプションについて理解することができた.
オプションを指定することで,動作が変わることに驚いた.
今までは,\verb|ls -a|など,オプションも含めて1つのコマンドだと考えていた.
端末でコマンドを扱えるようになったことが嬉しく,もっと深く理解したくなった.
自分でも詳しく調べてみようと思う.

%==================================================================%
\section{まとめ}
本レポートでは,CentOSの端末下におけるいくつかのをコマンドをまとめた。またいくつかのオプションについては,端末から\verb|--help|を用いて確認した.
%==================================================================%

\end{document}
