\documentclass[a4j,11pt]{jarticle}
% ファイル先頭から\begin{document}までの内容(プレアンブル)については,
% 教員からの指示がない限り, { } の中を書き換えるだけでよい.

% この部分は変更しないこと!
\usepackage[top=25truemm,  bottom=30truemm,
            left=25truemm, right=25truemm]{geometry}

% ToDo: 提出要領に従って,適切なタイトル・サブタイトルを設定する
% 「第 X 回」の表記は作成の都度変更すること.
\title{工学基礎実験実習 \\
       ファイル操作とシェル 第 1 回レポート}

% ToDo: 自分自身の氏名と学生番号に書き換える
\author{氏名: 重近 大智 (SHIGECHIKA, Daichi) \\
        学生番号: 09501527}

% ToDo: 教員の指示に従って適切に書き換える
\date{出題日: 2019年5月14日 \\
      提出日: 2019年5月日 \\
      締切日: 2019年5月21日 \\}  % 注:最後の\\は不要に見えるが必要.

% ToDo: 図を入れる場合,以下の1行を有効にする
%\usepackage{graphicx}

\begin{document}
\maketitle
% 目次つきの表紙ページにする場合はコメントを外す
%{\footnotesize \tableofcontents \newpage}

%==================================================================%
\section{はじめに}
CentOSの端末下におけるいくつかのコマンドの動作をまとめる.
引数についても記述する.
%==================================================================%
\section{ディレクトリ操作のコマンド}

%------------------------------------------------------------------%
\subsection{\texttt{ls}コマンド}
% 参考:sectionの中では verb が使えない.そんな時は?

\verb|ls|コマンドの概要は,以下の通りである.
\begin{description}
  \item[機能] % コマンドの機能内容を書く.
    カレントディレクトリ内のファイル一覧を表示する.
  \item[形式] % コマンドの入力形式を書く.
    \verb|ls| \verb|(option)| \verb|[directory name]|
  \item[オプション] % オプションと内容を箇条書きで書く.
    オプションは下記の通りである.
    \begin{itemize}
      \item \verb|-l|:  ファイルの詳細を表示する.
      \item \verb|-a|:  . と .. を含めて表示する.
      \item etc... //他のオプションを以下に箇条書きで書く.
    \end{itemize}
  \item[使用例] % コマンドの実行時に得られる結果を示す.
    \begin{verbatim}
    $ ls -la
      合計 20
      drwxr-xr-x 2 gotoh staff 4096 2014-04-26 11:40 .
      drwxr-xr-x 8 gotoh staff 4096 2007-12-01 22:26 ..
      -rw-r--r-- 1 gotoh staff 6671 2014-04-26 11:40 index.html
      -rw-r--r-- 1 gotoh staff 2193 2014-04-26 10:18 index.html~
      $
    \end{verbatim}
\end{description}

%------------------------------------------------------------------%
\subsection{\texttt{rm}コマンド}
\verb|rm|コマンドの概要は,以下の通りである.
\begin{description}
  \item[機能] % コマンドの機能内容を書く.
    ファイルを削除する.
  \item[形式] % コマンドの入力形式を書く.
    \verb|rm| \verb|(option)| \verb|[directory name]|
  \item[オプション] % オプションと内容を箇条書きで書く.
    オプションは下記の通りである.
    \begin{itemize}
      \item \verb|-i|:  削除前に確認メッセージを表示する.
      \item \verb|-r|:  ディレクトリとその中身も含めて削除する.
      \item \verb|-dir|: 空のディレクトリを削除する.
    \end{itemize}
  \item[使用例] % コマンドの実行時に得られる結果を示す.
    \begin{verbatim}
    $ rm -r 1
      $
    \end{verbatim}
\end{description}

\subsection{\texttt{pwd}コマンド}
\verb|pwd|コマンドの概要は,以下の通りである.
\begin{description}
  \item[機能] % コマンドの機能内容を書く.
    カレントディレクトリを表示する.
  \item[形式] % コマンドの入力形式を書く.
    \verb|pwd|
  \item[オプション] % オプションと内容を箇条書きで書く.
    オプションは見つからなかった.
  \item[使用例] % コマンドの実行時に得られる結果を示す.
    \begin{verbatim}
    $ pwd
  /home/users/ecs/09501527
      $
    \end{verbatim}
\end{description}

\subsection{\texttt{mkdir}コマンド}
\verb|mkdir|コマンドの概要は,以下の通りである.
\begin{description}
  \item[機能] % コマンドの機能内容を書く.
    ディレクトリを作成する.
  \item[形式] % コマンドの入力形式を書く.
    \verb|mkdir| \verb|(option)| \verb|[directory name]|
  \item[オプション] % オプションと内容を箇条書きで書く.
    \begin{itemize}
      \item \verb|-m|:  ファイルモードを設定する.
    \end{itemize}
  \item[使用例] % コマンドの実行時に得られる結果を示す.
    \begin{verbatim}
    $ mkdir 1
      $
    \end{verbatim}
\end{description}

\subsection{\texttt{rmdir}コマンド}
\verb|rmdir|コマンドの概要は,以下の通りである.
\begin{description}
  \item[機能] % コマンドの機能内容を書く.
    ディレクトリを削除する.
  \item[形式] % コマンドの入力形式を書く.
    \verb|rmdir| \verb|(option)| \verb|[directory name]|
  \item[オプション] % オプションと内容を箇条書きで書く.
    \begin{itemize}
      \item \verb|--ignore-fail-on-non-empty|:  ディレクトリが空でないため削除に失敗した場合、エラーだけを無視する.
    \end{itemize}
  \item[使用例] % コマンドの実行時に得られる結果を示す.
    \begin{verbatim}
    $ rmdir 1
      $
    \end{verbatim}
\end{description}

\subsection{\texttt{cd}コマンド}
\verb|cd|コマンドの概要は,以下の通りである.
\begin{description}
  \item[機能] % コマンドの機能内容を書く.
    カレントディレクトリを変更する.
  \item[形式] % コマンドの入力形式を書く.
    \verb|cd| \verb|[directory name]|
  \item[オプション] % オプションと内容を箇条書きで書く.
    \verb|[directory name]|を入力しなければ、ホームディレクトリに戻る.
  \item[使用例] % コマンドの実行時に得られる結果を示す.
    \begin{verbatim}
    $ cd p1
      $
    \end{verbatim}
\end{description}
%------------------------------------------------------------------%==================================================================%
\section{ファイル操作のコマンド}

%-----------------------------------------------------------------%
\subsection{\texttt{cat}コマンド}
\verb|cat|コマンドの概要は,以下の通りである.
\begin{description}
  \item[機能] % コマンドの機能内容を書く.
    ファイル内容のを連結したり、表示したりする.
  \item[形式] % コマンドの入力形式を書く.
    \verb|cat| \verb|(option)| \verb|[directory name]|
  \item[オプション] % オプションと内容を箇条書きで書く.
オプションは下記の通りである.
 \begin{itemize}
      \item \verb|-n|:  行番号を付けて表示する.
      \item \verb|-v|:  制御コードなどを含むファイルを表示する時に指定する.
      \item \verb|-b|:  空行を除いて行番号を付け加える. \verb|-n|より優先される.
    \end{itemize}
  \item[使用例] % コマンドの実行時に得られる結果を示す.
    \begin{verbatim}
$ cat -n bindec.c
     1	main(){
     2	  int i,j,k,l,p=0;
     3	  printf("2進数\t10進数\n");
     4	  for(i=0;i<2;i++)
     5	    for(j=0;j<2;j++)
     6	      for(k=0;k<2;k++)
     7		for(l=0;l<2;l++){
     8		  printf("%d%d%d%d\t%d\n",i,j,k,l,p++);
     9		}
    10	}

    \end{verbatim}
\end{description}

\subsection{\texttt{less}コマンド}
\verb|less|コマンドの概要は,以下の通りである.
\begin{description}
  \item[機能] % コマンドの機能内容を書く.
    ファイル内容を1画面ごとに表示する.
  \item[形式] % コマンドの入力形式を書く.
    \verb|cat| \verb|(option)| \verb|[directory name]|
  \item[オプション] % オプションと内容を箇条書きで書く.
オプションは下記の通りである.
 \begin{itemize}
      \item \verb|-n|:  行番号を付けて表示する.
      \item \verb|-v|:  制御コードなどを含むファイルを表示する時に指定する.
      \item \verb|-b|:  空行を除いて行番号を付け加える. \verb|-n|より優先される.
    \end{itemize}
  \item[使用例] % コマンドの実行時に得られる結果を示す.
    \begin{verbatim}
$ cat -n bindec.c
     1	main(){
     2	  int i,j,k,l,p=0;
     3	  printf("2進数\t10進数\n");
     4	  for(i=0;i<2;i++)
     5	    for(j=0;j<2;j++)
     6	      for(k=0;k<2;k++)
     7		for(l=0;l<2;l++){
     8		  printf("%d%d%d%d\t%d\n",i,j,k,l,p++);
     9		}
    10	}

    \end{verbatim}
\end{description}



3.2節以降,\verb|less|, \verb|mv|, \verb|cp|,などを節ごとに記載
%------------------------------------------------------------------


%==================================================================%
\section{考察}
本レポートで気づいたこと,感じたことを記載する.
「はじめに」と同様に,一文で終わらないように,文章をある程度書くこと.
この部分のみ,主観的に書いて良い.

%==================================================================%
\section{まとめ}
2章以降の内容に基づいて,本レポートで何をやったかを文章でまとめて下さい.
「はじめに」と同様に,一文で終わらないように,文章をある程度書くこと.
%==================================================================%

\end{document}
